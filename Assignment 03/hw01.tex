\documentclass[11pt, a4paper, fleqn]{article}
\usepackage[left=0.5in, top=0.5in, bottom=0.5in, right=0.5in]{geometry}
\usepackage{amsmath}
\usepackage{amsfonts}
\usepackage{setspace}

\begin{document}
\title{Homework Assignment 3\\
Harshita Agarwala}
\maketitle
\section{Problem 1}
Let the function be denoted by g($\theta$)
\begin{equation*}
g(\theta)=\theta^t (1-\theta)^h
\end{equation*}
Differentitating w.r.t to $\theta$ we get the first derivative
\begin{equation*}
g'(\theta)=t \theta^\text{t-1} (1-\theta)^h - h \theta^t (1 - \theta)^\text{h-1}
\end{equation*}
Differentitating w.r.t to $\theta$ again, we get the second derivate:
\begin{equation*}
g''(\theta)=t(t-1) \theta^\text{t-2} (1-\theta)^h -2th \theta^\text{t-1} (1 - \theta)^\text{h-1} + h(h-1) \theta^t (1 - \theta)^\text{h-2}
\end{equation*}
\\
Now we let f($\theta$)=log g($\theta$) and find the 1st and 2nd derivatives
\begin{equation*}
f(\theta) = log (g(\theta)) = \text{t log($\theta$) + h log(1-$\theta$)}
\end{equation*}
\begin{equation*}
f'(\theta) = 
\frac{\substack{t}}{\substack{\theta}} - \frac{\substack{h}}{\substack{1 - \theta}} 
\end{equation*}
\begin{equation*}
\text{and  }
f''(\theta) = 
\frac{\substack{h}}{\substack{(1 - \theta)^2}} - \frac{\substack{t}}{\substack{\theta^2}}
\end{equation*}


\section{Problem 2}
We first find critical points on f($\theta$) = log g($\theta$) by equating f'($\theta$) to 0
\begin{equation*}
\Rightarrow \frac{\substack{t}}{\substack{\theta}} - \frac{\substack{h}}{\substack{1 - \theta}} = 0
\end{equation*}
\begin{equation*}
\Rightarrow \frac{\substack{t - (t+h)\theta}}{\substack{\theta (1-\theta}} = 0
\end{equation*}
\begin{equation*}
\Rightarrow t-(t+h)\theta = 0
\end{equation*}
\begin{equation}
\Rightarrow \theta = \frac{\substack{t}}{\substack{t+h}}
\end{equation}
We substitute the value of $\theta$ in f''($\theta$) with that in equation (1) and get
\begin{equation*}
f''(\theta)= \frac{\substack{(t+h)^2}}{\substack{h}} - \frac{\substack{(t+h)^2}}{\substack{t}} = 
\frac{\substack{t-h}}{\substack{th}} (t+h)^2 < 0          \text{     if h $>$ t}
\end{equation*}
Again replacing the value of $\theta$ in g'($\theta$) with that in equation (1) we get
\begin{equation*}
g'(\theta)=t \Big[\frac{\substack{t}}{\substack{t+h}}\Big]^\text{t-1} \Big [ 1 - \frac{\substack{t}}{\substack{t+h}} \Big]^h
- h \Big[\frac{\substack{t}}{\substack{t+h}}\Big]^\text{t} \Big [ 1 - \frac{\substack{t}}{\substack{t+h}} \Big]^\text{h-1}
\end{equation*}
\begin{equation}
g'(\theta)=\frac{\substack{t^t h^h}}{\substack{(t+h)^\text{t+h-1}}} - \frac{\substack{t^t h^h}}{\substack{(t+h)^\text{t+h-1}}} = 0
\end{equation}
Hence, the critical points are same.\\
Now we find if it is a maximum or not.
\newpage
Now g''(($\theta$) is given by:
\begin{equation*}
g''(\theta)=t(t-1) \theta^\text{t-2} (1-\theta)^h -2th \theta^\text{t-1} (1 - \theta)^\text{h-1} + h(h-1) \theta^t (1 - \theta)^\text{h-2}
\end{equation*}
\begin{equation*}
g''\Big(\frac{\substack{t}}{\substack{t+h}}\Big)=
t(t-1) \Big(\frac{\substack{t}}{\substack{t+h}}\Big)^\text{t-2} \Big(1-\frac{\substack{t}}{\substack{t+h}}\Big)^h 
-2th \Big(\frac{\substack{t}}{\substack{t+h}}\Big)^\text{t-1} \Big(1-\frac{\substack{t}}{\substack{t+h}}\Big)^\text{h-1} 
+ h(h-1) \Big(\frac{\substack{t}}{\substack{t+h}}\Big)^t \Big(1-\frac{\substack{t}}{\substack{t+h}}\Big)^\text{h-2}
\end{equation*}
On solving further we get,
\begin{equation}
g''\Big(\frac{\substack{t}}{\substack{t+h}}\Big)=
- \frac{\substack{t^\text{t-1} h^\text{h-1}}}{\substack{(t+h)^\text{t+h-1}}} < 0 \text{  as t,h $\in$ $\mathbb{N}$}
\end{equation}
Therefore, log function retains the critical points of the main function

\section{Problem 3}
We know that $\theta_\text{MAP}$ is the maximum of the p($\theta$ = x $|$ D) and is given by':
\begin{equation*}
\theta_\text{MAP} =  \frac{\substack{N_T + a -1}}{\substack{N + a+b-1}} 
\text{   and   } 
\theta_\text{MLE} =  \frac{\substack{N_T}}{\substack{N}}
\end{equation*}
If $\theta_\text{MAP}$ = $\theta_\text{MLE}$ then a=b=1\\
This means that the prior distribution is uniform i.e there exists a prior p($\theta$) such that the result holds. Also, we know that such a prior will always exist.
\\Hence, it is true that $\theta_\text{MLE}$ is a special case of $\theta_\text{MAP}$

\section{Problem 4}
\begin{equation*}
\text{Now } \theta_\text{MLE} =  \frac{\substack{m}}{\substack{m+l}}
\end{equation*}
\begin{equation*}
\text{and } E_\text{PR}[\theta|a,b] =  \frac{\substack{a}}{\substack{a+b}}
\end{equation*}
The consequent posterior distribution is given by the Beta distribution Beta(x $|$ a+m, b+l) and the mean of this is:
\begin{equation*}
E_\text{PS}[X] =  \frac{\substack{a+m}}{\substack{a+b+m+l}} = 
\frac{\substack{a}}{\substack{a+b+m+l}} + \frac{\substack{m}}{\substack{a+b+m+l}}
\end{equation*}
\begin{equation*}
\text{Let   } 0 \leq \lambda \leq 1 \text{   be   }
\frac{\substack{a+b}}{\substack{a+b+m+l}}
\end{equation*}
Then we get, 
\begin{equation}
\lambda E_\text{PR} = \Big[\frac{\substack{a+b}}{\substack{a+b+m+l}}\Big]  \frac{\substack{a}}{\substack{a+b}} 
= \frac{\substack{a}}{\substack{a+b+m+l}}
\end{equation}

\begin{equation}
\text{and  } (1-\lambda) \theta_\text{MLE} = \Big[\frac{\substack{m+l}}{\substack{a+b+m+l}}\Big]  \frac{\substack{m}}{\substack{m+l}} 
= \frac{\substack{m}}{\substack{a+b+m+l}}
\end{equation}
On adding equation (4) and (5) we have, 
\begin{equation*}
 \lambda E_\text{PR} + (1-\lambda)\theta_\text{MLE} = E_\text{PS}
\end{equation*}
\newpage
\section{Problem 5}
The Poisson Distribution of X is given by
\begin{equation*}
P(X=k | \lambda) = \frac{\substack{\lambda^k e^\text{-$\lambda$}}}{\substack{k!}}
\end{equation*}
Now, for n i.i.d samples for X the probability is given by:
\begin{equation*}
P(D | \lambda) = \prod_{i=1}^{n}{\frac{\substack{\lambda^{k_i} e^\text{-$\lambda$}}}{\substack{k_i!}}}
\end{equation*}
\begin{equation}
\Rightarrow P(D | \lambda) =e^\text{-n$\lambda$} \prod_{i=1}^{n}\frac{\substack{\lambda^{k_i}}}{\substack{k_i!}}
\end{equation}
Taking log of equation (6) and denoting it by f($\lambda$) we have,

\begin{equation*}
f(\lambda) =-n\lambda + \Big(\sum_{i=1}^{n}{k_i}\Big) \log\lambda + \log\Big(\sum_{i=1}^{n}{k_i}\Big)
\end{equation*}
Differentiating w.r.t to $\lambda$ and equating to 0 we get
\begin{equation*}
f'(\lambda) =-n + \frac{\substack{\sum_{i=1}^{n}{k_i}}}{\substack{\lambda}} = 0
\end{equation*}
\begin{equation*}
\Rightarrow \lambda = \frac{\substack{\sum_{i=1}^{n}{k_i}}}{\substack{n}}
\end{equation*}
\begin{equation*}
\text{Therefore we have  }
\theta_\text{MLE} = \frac{\substack{\sum_{i=1}^{n}{k_i}}}{\substack{n}}
\end{equation*}
\\
Now we let the prior have a Gamma distribution with constants $\alpha$ and $\beta$
\begin{equation*}
P(\lambda|\alpha,\beta) = \frac{\substack{\beta^\alpha }}{\substack{\Gamma{\alpha}}} \lambda^\text{$\alpha$ -1} e^\text{$-\lambda\beta$}
\end{equation*}
Now the posterior distribution is given by:

\begin{equation*}
P(\lambda|D) = \frac{\substack{P(D|\lambda).P(\lambda) }}{\substack{P(D)}} 
\end{equation*}
On replacing values we get,

\begin{equation*}
P(\lambda|D) = \frac{\substack{1 }}{\substack{P(D)}} . 
\frac{\substack{e^\text{-n$\lambda$} \lambda^\text{$\sum{k_i}$}}}{\substack{\prod{k_i}}} . 
\frac{\substack{\beta^\alpha }}{\substack{\Gamma{\alpha}}} \lambda^\text{$\alpha$ -1} e^\text{$-\lambda\beta$}
\end{equation*}
\begin{equation*}
\Rightarrow 
P(\lambda|D) = c \Big[e^\text{(-n + $\beta$)$\lambda$}. \lambda^\text{($\alpha$ -1 + $\sum{k_i}$)}\Big]
\text{  where c is a constant}
\end{equation*}
Again taking log of the above function we get,
\begin{equation*}
(-n + \beta)\lambda + (\alpha -1 + \sum{k_i}) \log \lambda
\end{equation*}
 On finding the derivative and equating it to 0 we get,

\begin{equation*}
\lambda=\frac{\substack{\alpha -1 + \sum{k_i}}}{\substack{(-n + \beta)}} = \theta_\text{MAP}
\end{equation*}













\end{document}