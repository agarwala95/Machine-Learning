\documentclass[11pt, a4paper, fleqn]{article}
\usepackage[left=0.5in, top=0.5in, bottom=0.5in, right=0.5in]{geometry}
\usepackage{amsmath}
\usepackage{amsfonts}
\usepackage{setspace}

\begin{document}
\title{Machine Learning: Assignment 5\\
Harshita Agarwala}
\maketitle
\begin{singlespacing}
\section{Problem 1}
If we have CoX $\cap$ CoY $\neq \phi$ then $\exists$ atleast one point $\boldsymbol{t}$ $\in$ CoX and  $\boldsymbol{t}$ $\in$ CoY  
\\
$\Rightarrow$ $\exists$ $\alpha_i$s and $\beta_j$s such that
\begin{equation}
t = \sum_{i=1}^{N}{x_i \alpha_i}, \text{  } \alpha_i \geq 0 \text{  and  } \sum_{i=1}^{N}{\alpha_i} =1
\ \ \ \ \text{and} \ \ \ \
t = \sum_{j=1}^{M}{y_j \beta_j}, \text{  } \beta_j \geq 0 \text{  and  } \sum_{j=1}^{M}{\beta_j} =1
\end{equation}
If possible let there exist w and $w_o$  such that
\begin{equation*}
w^T x_i +w_o > 0  \  \forall \  i = 1, ... , N \text{  and   } w^T y_j + w_o < 0  \ \forall \ j = 1, ... , M
\end{equation*}
Now as  $\alpha_i$ $\geq$ 0 we have,
\begin{equation*}
\alpha_i(w^T x_i +w_o) \geq 0  \  \forall \  i = 1, ... , N 
\end{equation*}
Summing over all i we get,
\begin{equation*}
\sum{\alpha_i(w^T x_i +w_o)} > 0  \text{ as $ \sum_{i=1}^{N}{\alpha_i} =1 $}
\end{equation*}
\begin{equation}
\Rightarrow w^T \sum{\alpha_i x_i} +w_o \sum{\alpha_i} > 0  \ \ 
\Rightarrow w^T t +w_o > 0
\end{equation}
Similarly, we get
\begin{equation}
\sum{\beta_j(w^T y_j +w_o)} < 0 \ 
\Rightarrow w^T \sum{\beta_j y_j} +w_o \sum{\beta_j} < 0  \ \ 
\Rightarrow w^T t +w_o < 0
\end{equation}
Clearly, statements 2 and 3 contradict each other. Hence, there does not exist any w and $w_o$ such that the two sets are linearly separable.

\section{Problem 2}
We have to minimize the $E_w$(w) function i.e. equivalent to maximizing 
\begin{equation*}
\ln {\Big[\prod_{i=1}^{N}{\sigma(w^T x_i)^\text{$y_i$}(1 - \sigma(w^T x_i))^\text{1-$y_i$}}\Big] }
\end{equation*}
Now as $0 \leq \sigma(a) \leq 1 $, we would want the $\sigma(w^T x_i)$ and $(1 - \sigma(w^T x_i))$ to be 1.
\begin{equation*}
\text{for  } \sigma(w^T x_i) = \frac{\substack{1}}{\substack{exp(-w^T x_i) +1}} \rightarrow 1 
\end{equation*}
we will have $w \rightarrow \infty$ . Hence, we get
\begin{equation}
 \lim_{w\to\infty} \sigma(w^T x_i) = 1
\end{equation}
\begin{equation*}
\text{Similarly for  } \ (1-  \sigma(w^T x_i)) = 1 - \frac{\substack{1}}{\substack{exp(-w^T x_i) +1}} \rightarrow 1 
\end{equation*}
we will have $w \rightarrow -\infty$ . Hence, we get
\begin{equation}
 \lim_{w\to -\infty} 1 - \sigma(w^T x_i) = 1
\end{equation}
Therefore, we have the required result. Also, to penalize the high magnitudes of w, we add an error term like in the ridge regression.
\begin{equation*}
E(w) = -\ln {\Big[\prod_{i=1}^{N}{\sigma(w^T x_i)^\text{$y_i$}(1 - \sigma(w^T x_i))^\text{1-$y_i$}}\Big] } + \lambda||w||^2_q
\end{equation*}

\section{Problem 3}
We could use the basis function $\Phi(x_1,x_2) = \frac{x_2}{x_1}$ . This would make the circles transform to positive values and crosses to negative values.

\section{Problem 4}
The two points on the hyperplane are (0,5) and (2,0). As the data is two-dimensional, the hyperplane would be a line. Let the equation of the line be:
\begin{equation*}
w_0+w_1 x_1+w_2 x_2 =0 
\end{equation*}
On replacing the values (0,5) and (2,0) we get, 
\begin{equation*}
w_0 = -5w_2 \ \text{and} \ w_0 = -2w_1 \ \ \Rightarrow w_1 = \frac{5}{2}w_2
\end{equation*}
Taking $w_2$ = 1, the equation of the line is:
\begin{equation*}
-5 + \frac{5}{2} x_1+ x_2 =0 
\end{equation*}
\end{singlespacing}
\end{document}
