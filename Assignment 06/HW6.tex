\documentclass[10pt, a4paper, fleqn]{article}
\usepackage[left=0.5in, top=0.5in, bottom=0.5in, right=0.5in]{geometry}
\usepackage{amsmath}
\usepackage{amsfonts}
\usepackage{setspace}

\begin{document}
\title{Machine Learning: Assignment 6\\
Harshita Agarwala}
\maketitle
\begin{singlespacing}
\section{Problem 1}
i)\\
Now we know that x is both concave and convex. Therefore 3x would also be convex. Similarly $e^\text{y+z}$ is also convex. We can convert the minimum function to a maximum by using max(x) = -min(-x). Hence $-min(-x^2, \log(y))$ becomes $max(x^2, -\log(y))$ which is concave. Therefore f(x,y,z) is concave.\\ \\
ii)\\
We find the Hessian matrix for the given function. The required partial derivatives are:
\begin{equation}
\frac{{\partial f^2}}{{\partial x^2}} = 6yx - 4y \ \ \
\frac{{\partial f^2}}{{\partial x \partial y}} = 3 x^2 - 4 x\ \ \ 
\frac{{\partial f^2}}{{\partial y^2}} = 0 \ \ \ 
\frac{{\partial f^2}}{{\partial y \partial x}} = 3 x^2 - 4 x
\end{equation}
Substituting these values in the Hessian matrix H, we get:
\[\begin{bmatrix}
6yx - 4y&3 x^2 - 4 x\\
3 x^2 - 4 x&0
\end{bmatrix}\]
Now for H to be positive semidefinite, $\forall$ w $\in \mathbb{R}^2$ the product $wHw^T \geq 0$ . Let the vector be [1, 1]. Then $wHw^T$ is:
\begin{equation*}
6yx - 4y + 2(3 x^2 - 4 x)
\end{equation*}
As x,y $\in (-10,10)$, we pick any value of x, y and show that the product is $<$ 0. \\
On taking, x=8 and y=-9 we get, the product to be -76. Hence the Hessian matrix is not positive semidefinite. Therefore the f(x,y) is not convex.\\ \\
iii)\\
We find the second order derivative of the function.
\begin{equation*}
f(x) = \log{x} + x^3 \ \ 
f'(x) = \frac{1}{x} + 3 x^2 \ \ 
f''(x) = - \frac{1}{x^2} + 6 x \ \  
\end{equation*}
Now as the domain of the function is $(1,\infty)$ we have $- \frac{1}{\substack{x^2}} < 6x$ . Therefore as $f''(x) \geq 0$ the function is convex.\\ \\
iv)\\
We again convert the minimum function to a maximum one. f(x) is then equivalent to $max(-2\log{2x}, x^2 - 4x + 32)$ \\ 
Now, $2x$ is convex as x is convex(also concave). Also, as logarithmic function is concave, hence $-2 * \log{2x}$ is convex.
Similarly, $-4 * x$ is convex and $x^2$ and the constant 32 are also convex. Therefore $x^2 - 4x + 32$ is convex.
\\Hence, f(x) is convex
\section{Problem 2}
Let $f_1(x)$ and $f_2(x)$ be convex functions. Then, by the definition of convexity we have,
\begin{equation}
\lambda f_1(x) + (1-\lambda)f_1(y) \geq f_1(\lambda x + (1-\lambda)y) \ \ \forall \ \ \lambda \in [0,1]
\end{equation}
\begin{equation}
\lambda f_2(x) + (1-\lambda)f_2(y) \geq f_2(\lambda x + (1-\lambda)y) \ \ \forall \ \ \lambda \in [0,1]
\end{equation}
Let $\lambda$ in equation (2) and (3) be the same, then on adding the two equations we get,
\begin{equation*}
\lambda (f_1(x) + f_2(x)) + (1-\lambda)(f_1(y) + f_2(y)) \geq  f_1(\lambda x + (1-\lambda)y) +  f_2(\lambda x + (1-\lambda)y) \ \ for \ \ \lambda \in [0,1]
\end{equation*}
\begin{equation*}
\Rightarrow \lambda h(x) + (1-\lambda)h(y) \geq  h(\lambda x + (1-\lambda)y)  \ \ for \ \ \lambda \in [0,1]
\end{equation*}
As $\lambda \in$ [0,1] is arbitrary, therefore the result holds for all $\lambda \in$ [0,1]. Hence h is also a convex function. i.e. sum of two convex functions is also convex.
\section{Problem 3}
We will disprove the given function using an example. \\
Let $f_1(x) = x$ and $f_2(x) = x^2$. Then the product, $g(x) = f_1(x).f_2(x) = x^3$ . Now we know that $x$ is a function that is convex and $x^2$ is also a convex function. However, $x^3$ is not a convex function. 
\\Therefore the product of two convex functions is not necessarily a convex function. 
\section{Problem 4}
Let $\theta^*$ be a local minimum and $\beta$ be the global minimum such that, $f(\beta) < f(\theta^*)$ and $\beta \neq \theta^*$. \\We will show via contradiction that $f(\beta) = f(\theta^*)$ and $\beta = \theta^*$. Now we already know that
\begin{equation}
f(\beta) < f(\theta^*)
\end{equation}
From the definition of convexity, we have 
\begin{equation}
\lambda f(\beta) + (1-\lambda)f(\theta^*) \geq f(\lambda \beta + (1-\lambda)\theta^*) \ \ \forall \ \ \lambda \in [0,1]
\end{equation}
As $\theta^*$ is the local minimum and the value $ f(\lambda \beta + (1-\lambda)\theta^*)$ is in the neighborhood of $f(\theta^*)$, \\therefore $ f(\lambda \beta + (1-\lambda)\theta^*) > f(\theta^*)$ .  Hence equation (5) becomes, 
\begin{equation*}
\lambda f(\beta) + (1-\lambda)f(\theta^*) > f(\theta^*)
\ \ 
\Rightarrow \lambda f(\beta) > \lambda f(\theta^*)
\end{equation*}
As $\lambda \geq 0$, therefore we have
\begin{equation}
f(\beta) > f(\theta^*)
\end{equation}
From equation (4) and (6), we come to a contradiction, hence $f(\beta) = f(\theta^*)$ and $\beta = \theta^*$ i.e a convex function has no local minimum, only a global minimum.
\end{singlespacing}
\end{document}
